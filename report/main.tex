% Setup - do not change
\documentclass[11pt]{article}
\usepackage[top=0.9in, left=0.9in, bottom=0.9in, right=0.9in]{geometry} 
\usepackage{parskip}
\usepackage[english]{babel}
\usepackage[utf8]{inputenc}
\usepackage{amsmath,amsthm,amssymb,graphicx,pdfpages,lipsum,hyperref}
\usepackage[none]{hyphenat}
\usepackage{csquotes}

\setlength\parindent{0pt}
%%%%%%%%%%%%%%%%%%%%%%%%%%%%%%%%%%%%%%%%%%%%%%%%%%%%%%%%%%%%%%%%%%%
% add other packages here if required

%% Bibliography are specified in this file. You can also choose inline bib style if you want to. But make sure your citation style is consistent (and proper)
% For more details on citation: https://library.unimelb.edu.au/recite
\usepackage[sorting = none]{biblatex}
\addbibresource{references.bib}

%%%%%%%%%%%%%%%%%%%%%%%%%%%%%%%%%%%%%%%%%%%%%%%%%%%%%%%%%%%%%%%%%%% the '%' symbol denotes comments

% Begin document creation
% DELETE THE \lipsum PLACEHOLDERS WHEN YOU BEGIN
\title{\textbf{Does Viral Infection Affect Taxi Service Reliance?} \\ 
MAST30034 Assignment 1}
\author{
Xavier Travers \\
Student ID: 1178369 \\
%% Replace the link with your github repo
% 1. Remember to escape underscore in the link.
% 2. Remember to include the commit you want to submit in the link
TODO: \href{https://github.com/MAST30034-Applied-Data-Science/mast30034\_p1\_template/tree/fd9f1dd17fdbcb5b119b70c93a22da8210d44fd7}{Github Repository}
}

\begin{document}
\maketitle

\section{Introduction}
% Link to a 30 min tutorial if you require revision: https://www.overleaf.com/learn/latex/Learn_LaTeX_in_30_minutes

Viral infection has been on everyone's mind in the past few years due to the COVID-19 pandemic.
With lockdowns and the fear of the spread of infection, 
it is a natural assumption that many people-facing industries such as hospitality or ride-hailing have suffered in demand/reliance.
This report aims to contribute to a body of works attempting to quantify the effect that the COVID-19 pandemic has had on the reliance on ride-hail/taxi services,
and compare any derived trends to those based on another major viral infection: Influenza.

Throughout this research project, 
I will be investigating the impact that the COVID-19 pandemic has had on two measures of reliance on taxi services:
usage frequency, and trip distance. 
First, the relationships between reliance statistics and COVID-19 case, hospitalization, and death rates measured in New York City are modelled and analysed.
Next, a comparison is made to any trends that appear between the selected taxi reliance statistics and Influenza case rates.
Finally, a regression model will be constructed using both COVID-19 and Influenza case rates to predict thing taxi trip counts.

\textbf{TODO: What datasets are used?}

\begin{itemize}
    \item The New York City Taxi and Limousine Commission (TLC) has provided a dataset of taxi service trips which captures information such as travel distance and general pickup/dropoff locations \cite{tlcdataset}. 
Throughout this report, this is referred to as the TLC dataset. 
    \item 
\end{itemize}
TLC dataset \cite{tlcdataset}
COVID-19 dataset

\textbf{TODO: What is the timeline?}

The timeline of data considered starts in July 2019 and ends December 2021.
This allows for comparisons at different stages of the pandemic (including a 6 month period before it happened).
It is also more difficult to analyse the trip records of high-frequency taxi services such as Uber or Lyft,
for which a dedicated dataset was started in February 2019.


Throughout this research project, 
I will be investigating the impact that the COVID-19 pandemic has had on two primary aspects of the use of different taxi services:
usage frequency, and trip distance. Specifically, I will be modelling relationships between case, hospitalization, and death rates measured in New York City over the span of the timeline.

% \LaTeX{} Have many caveats, you should search stack overflow for latex tips whenever you feel something looks bad, for instance:
% When `` quoting '', should be used instead of ". For example, ``test'' vs "test".

% % use \textbf{} for bold text and \textit{} for italic. 
% % \texttt{} creates code blocks akin to `code ticks` in markdown
% \textbf{Please refer to the spec, the word count and page count is strict.} Feel free to change the section headings (and we recommend you do).

% Always remember to cite materials that does not belong to you. For instance, you should cite the sensor datasets \cite{2022sensorreading, 2022sensorlocation}.
% % Example here used biblatex to manage citations: https://www.overleaf.com/learn/latex/Bibliography_management_with_biblatex , You are free to choose your own way for managing references if biblatex seems too hard.

% \lipsum[7]

% You can have \section{}, \subsection{}, and \subsubsection{}
\section{Method}

\subsection{Preprocessing}

\subsection{Analysis and Modelling}


% \begin{enumerate} 
%     \item Example for enumerated points
%     % use \item to create more points
% \end{enumerate}

% \begin{itemize} 
%     \item Example for dot points
%     \item[*] You can change dot points to any symbols by putting [SYMBOL].
%     \item[$\times$] Here's a fun example.
% \end{itemize} 
% \lipsum[4-5]
% Example code for figures:
% % the [h] ensures your figure is inline at the location and not displayed on some other page
% \begin{figure}[h]
%     % change the scale multiplier to make the figures smaller or larger
%     \includegraphics[width=0.35\textwidth]{example-image-a}
%     % this ensures your figures are centered where possible
%     \centering
%     \caption{Some caption} % refer to this image as (Figure 1)
% \end{figure}
% \lipsum[1-2]

% Example of a maths equation:
% \begin{equation}
%     Y = X\beta + \epsilon
% \end{equation}

% Example of an aligned equation (\& denotes the symbol to align):
% \begin{align*}
%     E[\mathbf{y}] &= X\beta + E[0] \\
%                   &= X\beta
% \end{align*}

% Example of an in-line equation $\epsilon \sim N(0, 1)$ \\

\subsection{Geospatial Visualisation}

\section{Recommendations}
% \lipsum[10]

\section{Conclusions}
% \lipsum[14-15]


\clearpage

% BEGIN REFERENCES SECTION
\printbibliography

\end{document}