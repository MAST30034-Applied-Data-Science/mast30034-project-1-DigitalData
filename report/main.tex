% Setup - do not change
\documentclass[11pt]{article}
\usepackage[top=0.9in, left=0.9in, bottom=0.9in, right=0.9in]{geometry} 
\usepackage{parskip}
\usepackage[english]{babel}
\usepackage[utf8]{inputenc}
\usepackage{amsmath,amsthm,amssymb,graphicx,pdfpages,lipsum,hyperref}
\usepackage[none]{hyphenat}
\usepackage{csquotes}

\setlength\parindent{0pt}
%%%%%%%%%%%%%%%%%%%%%%%%%%%%%%%%%%%%%%%%%%%%%%%%%%%%%%%%%%%%%%%%%%%
% add other packages here if required

%% Bibliography are specified in this file. You can also choose inline bib style if you want to. But make sure your citation style is consistent (and proper)
% For more details on citation: https://library.unimelb.edu.au/recite
\usepackage[sorting = none]{biblatex}
\addbibresource{references.bib}

%%%%%%%%%%%%%%%%%%%%%%%%%%%%%%%%%%%%%%%%%%%%%%%%%%%%%%%%%%%%%%%%%%% the '%' symbol denotes comments

% Begin document creation
% DELETE THE \lipsum PLACEHOLDERS WHEN YOU BEGIN
\title{\textbf{How Does Viral Infection Affect Taxi Service Reliance?} \\ 
MAST30034 Assignment 1}
\author{
Xavier Travers \\
Student ID: 1178369 \\
%% Replace the link with your github repo
% 1. Remember to escape underscore in the link.
% 2. Remember to include the commit you want to submit in the link
TODO: \href{https://github.com/MAST30034-Applied-Data-Science/mast30034\_p1\_template/tree/fd9f1dd17fdbcb5b119b70c93a22da8210d44fd7}{Github Repository}
}

\begin{document}
\maketitle

\section{Introduction}
% Link to a 30 min tutorial if you require revision: https://www.overleaf.com/learn/latex/Learn_LaTeX_in_30_minutes

Viral infection has been on everyone's mind in the past few years due to the COVID-19 pandemic.
With lockdowns and fears of infection, 
it is a natural assumption that many people-facing industries such as ride-hailing have suffered in demand.
To what extent is such an assumption true?
This report aims to contribute to a body of works attempting to quantify the effect that larger-scale viral diseases can have on different industries.
Specifically investigated will be the effects that influenza/COVID-19 case rates have on taxi trip rates and passenger counts.

This report will detail the statistical analysis involved in revealing the extent to which changes in case rates of a viral infection may affect taxi trip rates and passenger counts.

\subsection{Datasets}
\begin{itemize}
    \item The New York City Taxi and Limousine Commission (TLC) provides a dataset of taxi service trips which captures information such as type of taxi, travel distance, general pickup/dropoff locations and driver-input passenger counts \cite{tlcdataset}. 
Throughout this report, this is referred to as the TLC dataset. 
In this report, the focus is placed on the dataset concerning New York's Yellow hail taxis. 
High frequency hired vehicle service (e.g. Uber) data is not included in analysis, 
since the dataset is too recent for the purposes of this report (starting only in early 2019).
Each entry in this dataset contains several values, but this report only focuses on trip distance, passenger count, date of pickup, and the pickup/dropoff locations.
    \item Influenza case rates are recorded on a weekly basis by the New York Department of Health \cite{fludataset}. 
Case rates in this dataset are dated based on Morbidity and Mortality Weekly Report (MMWR) weeks, which are generated using rules defined by the CDC \cite{mmwr}.
Each entry in this dataset contains an MMWR week, county (within the state of New York), type of Influenza (A, B or unspecified), and case count.
The dataset provides coverage over the whole selected timeline of data.
    \item COVID-19 case rates have been recorded daily by the New York Department of Health and Mental Hygiene \cite{coviddataset}.
This dataset begins on the last day of february, when the first official cases of COVID-19 were recorded in New York City. 
Each entry in this data set contains a date and several of the daily COVID-19 rates by borough (e.g. count of hospitalizations on the day in the Bronx).
Of specific interest is the daily case count per borough.
\end{itemize}

\subsection{Timeline}
The overall timeline considered for the data spans from the start of 2018 to the end of 2021.
This can be split into two shorter timelines of equal length: before COVID-19, and after COVID-19.
Data from 2022 is not included, since the COVID-19 dataset is automated and has an increased likelihood of containing false information with newer data.
Data from before 2018 is not included for the sake of symmetry and to reduce code runtime.

% The timeline of data considered starts in July 2019 and ends December 2021.
% This allows for comparisons at different stages of the pandemic (including a 6 month period before it happened).
% It is also more difficult to analyse the trip records of high-frequency taxi services such as Uber or Lyft,
% for which a dedicated dataset was started in February 2019.

% \LaTeX{} Have many caveats, you should search stack overflow for latex tips whenever you feel something looks bad, for instance:
% When `` quoting '', should be used instead of ". For example, ``test'' vs "test".

% % use \textbf{} for bold text and \textit{} for italic. 
% % \texttt{} creates code blocks akin to `code ticks` in markdown
% \textbf{Please refer to the spec, the word count and page count is strict.} Feel free to change the section headings (and we recommend you do).

% Always remember to cite materials that does not belong to you. For instance, you should cite the sensor datasets \cite{2022sensorreading, 2022sensorlocation}.
% % Example here used biblatex to manage citations: https://www.overleaf.com/learn/latex/Bibliography_management_with_biblatex , You are free to choose your own way for managing references if biblatex seems too hard.

% \lipsum[7]

% You can have \section{}, \subsection{}, and \subsubsection{}
\section{Method}

\subsection{Preprocessing}


\subsubsection{Cleaning}

\textbf{TLC Dataset}
\begin{enumerate}
    \item test
\end{enumerate}

\subsubsection{Necessary Extracted Features}
The flu dataset contains detail only on a weekly basis, 
while the other datasets used contain daily data. 
This means that the data needs to be aggregated by MMWR week value,
which allows the construction of linear models on weekly data.
While the granularity of the data suffers, 
the large timeline ensures that at least 52 points of data are available for analysis per time period.
Also necessary for connecting viral infection case rate datasets to the taxi datasets are the extraction of borough.

\subsubsection{Aggregation}

All data is aggregated by MMWR week.
The datasets are then all joined on the MMWR week, and corresponding boroughs.
Importantly, for each MMWR week, two sets of case rates are recorded: 
those corresponding to the pickup borough, and those corresponding to the dropoff borough.
For each MMWR week aggregate, total case rates are recorded, 
as well as total average trip distances and average passenger counts.

\subsection{Analysis and Modelling}


% \begin{enumerate} 
%     \item Example for enumerated points
%     % use \item to create more points
% \end{enumerate}

% \begin{itemize} 
%     \item Example for dot points
%     \item[*] You can change dot points to any symbols by putting [SYMBOL].
%     \item[$\times$] Here's a fun example.
% \end{itemize} 
% \lipsum[4-5]
% Example code for figures:
% % the [h] ensures your figure is inline at the location and not displayed on some other page
% \begin{figure}[h]
%     % change the scale multiplier to make the figures smaller or larger
%     \includegraphics[width=0.35\textwidth]{example-image-a}
%     % this ensures your figures are centered where possible
%     \centering
%     \caption{Some caption} % refer to this image as (Figure 1)
% \end{figure}
% \lipsum[1-2]

% Example of a maths equation:
% \begin{equation}
%     Y = X\beta + \epsilon
% \end{equation}

% Example of an aligned equation (\& denotes the symbol to align):
% \begin{align*}
%     E[\mathbf{y}] &= X\beta + E[0] \\
%                   &= X\beta
% \end{align*}

% Example of an in-line equation $\epsilon \sim N(0, 1)$ \\

\subsection{Geospatial Visualisation}

\section{Recommendations}
% \lipsum[10]

\section{Conclusions}
% \lipsum[14-15]


\clearpage

% BEGIN REFERENCES SECTION
\printbibliography

\end{document}